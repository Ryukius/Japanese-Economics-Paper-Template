\documentclass[12pt,a4paper]{ltjarticle}
% ===== LuaLaTeX(メイン)=====
\usepackage{unicode-math}
\usepackage{luatexja-fontspec}

% ローカルのBIZUDフォントを使用(推奨)
\setmainfont{Latin Modern Roman}
\setmathfont{Latin Modern Math}
\setmainjfont[BoldFont=BIZUDGothic-Bold]{BIZUDMincho-Regular}[Path=./fonts/]

% システムデフォルトフォントを使用したい場合は、上記3行をコメントアウトして以下を使用
% \setmainfont{Latin Modern Roman}
% \setmathfont{Latin Modern Math}
% \setmainjfont[BoldFont=IPAGothic]{IPAMincho}

% スタイルファイルのパスを設定
\makeatletter
\def\input@path{{styles/}}
\makeatother

\usepackage{natbib}

\usepackage{geometry}   % ページレイアウトの設定
\usepackage{ulem}       % 下線や取り消し線のサポート
\usepackage{graphicx}   % 画像の挿入および操作
\usepackage{caption}    % 図や表のキャプションのカスタマイズ
\usepackage{color}      % 文字や背景の色付け
\usepackage{setspace}   % 行間の設定
\usepackage{comment}    % コメントアウト用環境
\usepackage{footmisc}   % 脚注のカスタマイズ
\usepackage{subcaption} 
\usepackage{indentfirst}
% \usepackage{pdflscape}

\definecolor{MyBrown}{rgb}{0.3,0,0}
\definecolor{MyBlue}{rgb}{0,0,0.3}
\definecolor{MyRed}{rgb}{0.6,0,0.1}
\definecolor{MyGreen}{rgb}{0,0.4,0}

\geometry{left=1.0in,right=1.0in,top=1.0in,bottom=1.0in}

%% hyperref
\usepackage[%
bookmarks=true,%
bookmarksnumbered=true,%
colorlinks=true,%
linkcolor=MyBlue,%
citecolor=MyRed,%
filecolor=MyBlue,%
urlcolor=MyGreen%
]{hyperref}

\newcommand{\red}[1]{{\color{red} #1}}
\newcommand{\blue}[1]{{\color{blue} #1}}

\renewcommand{\abstractname}{要約}


\begin{document}

\begin{titlepage}
\title{タイトル\thanks{謝辞}}
\author{著者名\thanks{所属大学所属学部}}
\date{\today}
\maketitle
\begin{abstract}
 本要約は、特定の内容を説明するものではなく、あくまでテンプレートとしての役割を果たすことを目的としている。そのため、具体的な事例やデータ、分析結果は一切含まれておらず、あたかも意味がありそうに見える文章構成で構成されている。要点や結論らしきものが示されているように見えるが、実際には何も主張していない。適度に抽象的な表現を用いることで、どのようなテーマにも適用可能な汎用性を意図している。今後この要約は、テンプレートを用いて執筆する論文の内容に合わせ、適宜修正されることが期待される。\\
\vspace{0in}\\
\noindent\textbf{キーワード:} key1, key2, key3\\
\vspace{0in}\\
\noindent\textbf{JELコード:} key1, key2, key3\\

\bigskip
\end{abstract}
\setcounter{page}{0}
\thispagestyle{empty}
\end{titlepage}
\pagebreak \newpage

\section{序論}

test

\section{先行研究}

\section{モデル}

\section{データ}

\section{推定結果}

\section{結論}

\bibliographystyle{jecon}

\bibliography{bibliography}

\end{document}
