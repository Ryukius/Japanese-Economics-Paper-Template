% ===== LuaLaTeX(メイン)=====
\usepackage{unicode-math}
\usepackage{luatexja-fontspec}

% ローカルのBIZUDフォントを使用(推奨)
\setmainfont{Latin Modern Roman}
\setmathfont{Latin Modern Math}
\setmainjfont[BoldFont=BIZUDGothic-Bold]{BIZUDMincho-Regular}[Path=./fonts/]

% システムデフォルトフォントを使用したい場合は、上記3行をコメントアウトして以下を使用
% \setmainfont{Latin Modern Roman}
% \setmathfont{Latin Modern Math}
% \setmainjfont[BoldFont=IPAGothic]{IPAMincho}

% スタイルファイルのパスを設定
\makeatletter
\def\input@path{{styles/}}
\makeatother

\usepackage{natbib}

\usepackage{geometry}   % ページレイアウトの設定
\usepackage{ulem}       % 下線や取り消し線のサポート
\usepackage{graphicx}   % 画像の挿入および操作
\usepackage{caption}    % 図や表のキャプションのカスタマイズ
\usepackage{color}      % 文字や背景の色付け
\usepackage{setspace}   % 行間の設定
\usepackage{comment}    % コメントアウト用環境
\usepackage{footmisc}   % 脚注のカスタマイズ
\usepackage{subcaption} 
\usepackage{indentfirst}
% \usepackage{pdflscape}

\definecolor{MyBrown}{rgb}{0.3,0,0}
\definecolor{MyBlue}{rgb}{0,0,0.3}
\definecolor{MyRed}{rgb}{0.6,0,0.1}
\definecolor{MyGreen}{rgb}{0,0.4,0}

\geometry{left=1.0in,right=1.0in,top=1.0in,bottom=1.0in}

%% hyperref
\usepackage[%
bookmarks=true,%
bookmarksnumbered=true,%
colorlinks=true,%
linkcolor=MyBlue,%
citecolor=MyRed,%
filecolor=MyBlue,%
urlcolor=MyGreen%
]{hyperref}

\newcommand{\red}[1]{{\color{red} #1}}
\newcommand{\blue}[1]{{\color{blue} #1}}

\renewcommand{\abstractname}{要約}
